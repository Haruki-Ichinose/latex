\documentclass[dvipdfmx]{jsarticle}

% --- パッケージ ---
\usepackage[dvipdfmx]{graphicx}
\usepackage{url}
\usepackage{hyperref}
\usepackage{cite}
\usepackage{indentfirst}
\usepackage{titlesec}
\usepackage[top=5mm, bottom=7mm, left=13mm, right=13mm]{geometry} % さらに余白を狭く
\pagestyle{empty}
\usepackage{enumitem}

% --- レイアウト設定 ---
\setlength{\parindent}{1em}
\setlength{\parskip}{0pt}

% --- セクションタイトルの体裁 ---
\titleformat{\section}
  {\rmfamily\large\bfseries}{\thesection}{1em}{\bfseries}
\titlespacing*{\section}{0pt}{0.5em}{0.2em}

\titleformat{\subsection}
  {\rmfamily\normalsize\bfseries}{\thesubsection}{1em}{\bfseries}
\titlespacing*{\subsection}{0pt}{0.3em}{0.1em}

\titleformat{\subsubsection}
  {\rmfamily\normalsize\bfseries}{\thesubsubsection}{1em}{\bfseries}
\titlespacing*{\subsubsection}{0pt}{0.2em}{0.05em}

% --- タイトルを上に詰める ---
\makeatletter
\renewcommand{\maketitle}{
  \begin{center}
    {\Large\bfseries \@title \par}
    \vspace{0.1em}
    {\normalsize \@author \par}
  \end{center}
  \vspace{-0.5em}
}
\makeatother

\begin{document}

\title{これまでの研究活動や大学の講義や演習で行った実践的な活動}
\author{一瀬 遥希}
\date{}
\maketitle

\section{現在行っている研究活動 (研究テーマ:偏見的統計データ説明文に対する対抗文生成)}

\subsection{概要}
現在行っている研究では, 先行研究\cite{zhang2022}で定義された「人間の本能を悪用した偏見的統計データ説明文」に着目している.
これらの説明文に対し, VLMやLLMを活用することで, 偏見を是正し, 事実に基づいた客観的な情報を提供する対抗文(Counter-Narrative)を自動生成する手法の確立を目指している.

\subsection{アプローチと研究内容}
対抗文生成モデルの設計では, VLMを用いて統計グラフや画像から事実情報を抽出し, 先行研究での分類\cite{zhang2022,zhang2023}を基に説明文とデータの矛盾を客観的に特定する.
その後, LLMに抽出した事実情報と偏見的説明文が悪用する本能(例:ギャップ本能)パターンを入力し,
偏見を指摘しつつ客観的かつ中立的な言葉で統計データの全体像を提示する対抗文を生成するよう促す.
生成された対抗文の評価には, 人間評価とLLMによる自動評価の両方を検討している.

\subsection{期待される成果と自身の貢献}
本研究は, 統計データに基づく誤解を自動で是正し, 正確な情報理解を促す対抗文生成システムの構築を目指している.
これはデジタル時代の情報リテラシー向上とAI倫理への貢献が期待できる.

\section{学内における実践的な活動}

\subsection{プログラミングとデータ構造・アルゴリズムの実践, 機械学習とデータマイニングの演習}
大学の講義および演習を通じて, プログラミングの基礎から応用まで幅広く学んだ. Python, Scheme, C言語を用いた演習では, データ構造やアルゴリズムを実装し, 効率的なコード設計と問題解決能力を養った. また, 大規模データセットを用いた機械学習モデルの構築演習では, ニューラルネットワークの基礎, 学習プロセス, 評価方法を実践的に学び, 気象データなど実データを用いたデータマイニング演習では, 前処理, 統計分析, 可視化を通じて有用な洞察を抽出するプロセスを経験した.

\subsection{低レベルプログラミングとシステムプログラミング}
C言語およびアセンブリ言語を用いた簡易OSのチーム開発を経験した.
メモリ管理, プロセス管理, ファイルシステムなどコンピュータシステムの基幹部分を実装することで, ハードウェアとソフトウェアの連携, システムの動作原理を深く理解した.
この経験を通じて, システムの根幹を支える低レベルプログラミングの知識とスキル, チーム開発における役割分担やコードレビュー, バージョン管理の重要性など実践的なスキルを身につけた.

\section{学外における実践的な活動}

\subsection{データエンジニアカタパルトでの活動 (2023年8月~2024年2月)}
約7ヶ月間, 福岡市が主催するデータエンジニアカタパルトに参加し, 実務に近い形でWebフレームワーク開発とデータ可視化技術を習得した.
\begin{itemize}[itemsep=0pt,topsep=2pt,parsep=0pt,partopsep=0pt]
    \item \textbf{Webアプリケーション開発}:
          PHPのLaravelフレームワークを使用した講座を受講し, 実務で求められるWebアプリケーション開発の知識(MVCモデル, データベース連携, ルーティングなど)を体系的に習得した.
          その後, 個人開発プロジェクトにおいて, GitHubを活用したバージョン管理の下, 効率的な開発手法を実践した.
    \item \textbf{データ可視化とビジネスインテリジェンス}:
          Looker Studioを用いたデータ可視化技術を習得し, 収集したデータや分析結果を効果的に可視化し, ビジネス上の意思決定に役立てるスキルを身につけた.
    \item \textbf{企業課題への取り組み}:
          最終フェーズでは, 企業のメンターの指導のもと, 実際の企業課題(個人輸出品に対する自国及び相手国の関連書類の検索)に対し, 目的としたプロダクトの開発に携わった.
\end{itemize}

\subsection{個人開発・Kaggleコンペティションへの参加}
個人開発として, Pythonを用いた機械学習モデルの構築やReactを用いたWebアプリケーションの開発に取り組んでいる.
また, Kaggleコンペティションにも参加し, 様々なデータセットを用いた予測モデルの構築や評価手法の実践を通じて, データサイエンスの実務的なスキルを磨いている.

\begin{thebibliography}{99}
    \bibitem{zhang2022}
    K. Zhang, H. Shinden, T. Mutsuro, and E. Suzuki. Judging Instinct Exploitation in Statistical Data Explanations Based on Word Embedding. In \textit{Proc. AIES ’22}, pp. 867–879, 2022.

    \bibitem{zhang2023}
    K. Zhang and E. Suzuki. Judging Credible and Unethical Statistical Data Explanations via Phrase Similarity Graph. In \textit{Proc. 2023 Pacific Asia Conference on Information Systems (PACIS)}, 2023.
\end{thebibliography}
\end{document}