\documentclass[main.tex]{subfiles}

\begin{document}

\begin{center}
    {\Large 志望分野・志望理由・入学後の研究計画}\\[1em]
    九州大学工学部電気情報工学科計算機工学コース 4年\\
    一瀬 遥希
\end{center}
\vspace{1.5em}

\section*{1. 志望分野}
ここに志望分野を記述します。
(例:インタラクションデザイン、人工知能応用、バーチャルリアリティ技術など)

\subsection*{1.1 分野の現状と課題(もしあれば)}
その分野の現状や課題認識について触れると、より深みが増します。

\section*{2. 志望理由}
なぜ東京大学大学院情報理工学系研究科の創造情報学専攻を志望するのか。
\begin{itemize}
    \item これまでの学習や経験との関連
    \item 興味がある教員とその理由
    \item カリキュラムや研究環境の魅力
\end{itemize}

\section*{3. 入学後の研究計画}
\subsection*{3.1 研究の背景と目的}
必要性、目的、解明したいこと等を具体的に。

\subsection*{3.2 研究内容と手法}
どのような手法で進めるか具体的に。

\subsection*{3.3 期待される成果と意義}
得られる成果とその意義などをまとめる。

\subsection*{3.4 年次計画(任意)}
研究全体の大まかな進め方(例:修士2年間の計画)。

\end{document}