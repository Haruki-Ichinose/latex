\documentclass[dvipdfmx]{jarticle}
\usepackage{graphicx}
\usepackage{amsmath,amssymb}
\usepackage{bm}
\usepackage{float}
\usepackage{multirow}
\usepackage{caption}
\usepackage{subcaption}
\usepackage{color}
\usepackage[a4paper, margin=25mm]{geometry}

% 段落の間に1行の空白を入れる
\setlength{\parskip}{1\baselineskip plus 1pt minus 1pt}

% 日付を非表示にする
\date{}

\title{九州大学大学院システム情報科学府修士課程\\特別試験 小論文}
\author{九州大学工学部電気情報工学科\\一瀬 遥希}

\begin{document}
\maketitle

\section{現在行っている研究}
現在、私は「データの偏見的説明」に対する対抗説明文の生成に関する研究準備を進めています。
統計データの解釈には、しばしば人間の認知バイアスや、固定化された思考パターンが無意識のうちに影響を与え、倫理的に問題のある偏った説明が、一定の信頼性をもって受け手に信じられてしまうことがあります。
このような現象は、Roslingらの著書『Factfulness』で紹介された「ギャップ本能」や「宿命本能」など、人間の10の本能を通じて説明されており、これらの本能の悪用によって、根拠の乏しい説明が説得力をもって広まる危険性があります。

このような偏見的説明を自動的に判定するため、先行研究では、意味的類似度に基づいたフレーズ埋め込みや、類義語同士の関係性をグラフ構造として捉える手法などが提案されています。
さらに、偏見的説明の構造に着目し、それらを分類・モデル化した上で、大規模言語モデル(LLM)を用いて説明文を生成するアプローチも試みられています。

私は現在、こうした先行研究の精読と分析を通じて、自身の研究課題を明確化する段階にあります。
特に関心を持っているのは、時事ニュースなどに含まれる偏見的な説明文を分析し、それに対して統計的事実に基づく対抗説明文を生成する手法の構築です。
この対抗文生成を通じて、偏見的説明の構造や特徴を明らかにするとともに、受け手の誤解や先入観を和らげる情報提供のあり方を探究していきたいと考えています。

\section{入学後行ってみたい研究}
大学院進学後は、鈴木英之進教授のご指導のもと、現在の研究をさらに発展させたいと考えています。
具体的には、説明文に含まれる主語・性質・比較対象といった構成要素を抽出し、それらの意味的関係を定量的に評価することで、偏見的説明に共通する表現パターンを抽出・モデル化することを目指します。
また、自然言語処理と知識構造の融合によって、意味的なバイアスや印象操作の兆候を自動的に検出できる仕組みの構築にも挑戦したいと考えています。

加えて、機械学習の理論と応用についても体系的に学びたいと考えています。
特に、アルゴリズムの仕組みや数理的な背景を理解することで、機械学習を単なるツールとしてではなく、課題に応じて適切に設計・運用できる柔軟な思考力を身につけたいです。

また将来的には、研究活動に取り組みながら、エンジニアとして実務に従事することで、データの実践的な活用手法やチームでの開発手法を学び、情報技術者としてのスキルを身に着けたいと考えています。

\section{情報技術者・研究者としての自身の将来像}
私は将来、技術的な専門性と社会的な責任感を兼ね備えた情報技術者、特にシステムエンジニアやAIエンジニアとして、社会に貢献していきたいと考えています。
情報技術においては、精度や効率といった技術的性能だけでなく、「その技術がどのように人に受け取られ、社会にどのような影響を及ぼすか」という観点を踏まえた設計と運用が不可欠です。
とりわけ、AIや自動化技術が広く浸透する現代においては、情報の提示や説明のあり方が、受け手の理解や判断に大きな影響を与える場面が少なくありません。
私は、そうした状況において、情報の「信頼性」と「受容性」のバランスを意識しながら、人々の意思決定を支える情報提供の仕組みを技術面から支えていきたいと考えています。

また私は、社会にとって意味ある価値を創出できる技術者でありたいと考えています。
そのために、特定の分野にとどまらず、さまざまな領域に関心を持ち、自発的に学び続ける姿勢を大切にしながら知見を広げていきたいと思います。
そうして得られた知識や経験は社会に還元し、誰もが安心して利用できる情報環境の実現に貢献していきたいです。

私は、技術はそれ自体が目的ではなく、社会や人々の課題を解決するための手段だと捉えています。
今後も、技術を正しく使いこなしながら、現実社会の問題に向き合い、実践的な研究と開発に真摯に取り組んでいきたいと思います。

\end{document}